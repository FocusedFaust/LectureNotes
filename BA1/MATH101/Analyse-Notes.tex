\documentclass[10pt,a4paper]{book}
\usepackage[utf8]{inputenc}
\usepackage{amsmath}
\usepackage{amsfonts}
\usepackage{amssymb}
\usepackage{multicol}
\usepackage{hyperref}
\usepackage[ruled, lined, longend]{algorithm2e}
\usepackage[shortlabels]{enumitem}
\usepackage{textcomp}

\setlength{\parindent}{20pt}

\newcommand{\R}{\mathbb{R}}
\newcommand{\N}{\mathbb{N}}
\newcommand{\Z}{\mathbb{Z}}
\newcommand{\ind}{\hspace*{\parindent}}

\title{Analyse I - Notes et Résumés}
\author{Faustine Flicoteaux}
\date{Semestre d'automne 2024}

\begin{document}
\maketitle
\tableofcontents
\newpage


\section*{Introduction}
\addcontentsline{toc}{section}{\protect\numberline{}Introduction}
Ce qui suit est mes propres notes et explications. Évidemment, tout ça ne me vient pas par l'opération du Saint-Esprit, mais du cours de ma professeur, Prof. Anna Lachowska, et de vidéos et cours en ligne.\par 
J'explique principalement des concepts que je ne comprends pas totalement, c'est donc à prendre avec des pincettes. Si vous remarquez une erreur (mathématique ou de langue), n'hésitez pas à m'en faire part à mon adresse e-mail de l'EPFL \texttt{\href{mailto:faustine.flicoteaux@epfl.ch}{faustine.flicoteaux@epfl.ch}}.\par 
Le dernier fichier \LaTeX et le pdf correspondant peuvent être trouvés sur mon dépôt GitHub \url{https://github.com/FocusedFaust/LectureNotes}.

\chapter{Suites et séries}
\section{Limites à retenir}
\begin{itemize}
\item $\lim_{n\to\infty} a_0r^n = \left\{
        \begin{array}{ll}
            0 & \quad |r| < 1\\
            a_0 & \quad r = 1\\
            diverge & \quad |r| > 1 \text{ ou } r = -1
        \end{array}\right.$
\item $\lim_{n\to\infty} \sqrt[n]{a} = 1$ pour tout $a > 1$
\item $\lim_{n\to\infty} \dfrac{1}{n^p}$ pour tout $p > 0$
\item $\lim_{n\to\infty} \dfrac{sin(1/n)}{1/n} = 1$
\item $\lim_{n\to\infty} \sqrt{1-(\dfrac{1}{n})^2} = 1$
\item $\lim_{n\to\infty} \dfrac{p^n}{n!} = 0$ pour tout $p > 0$
\item $\lim_{n\to\infty} \dfrac{\alpha^n}{n\text{log}(n)} = +\infty$ pour $\alpha>1$ parce que $\alpha^n$ grandit plus vite que $n\text{log}(n)$
\end{itemize}

\section{Ordre de croissance de quelques fonctions}
L'ordre de croissance est utile lorsque nous étudions la limite à l'infini d'une fonction rationnelle. Si le dénominateur croît plus vite que le numérateur, la limite vaudra 0. Lorsque $x\to\infty$, on a donc :
\[\text{cos}(x)\prec\text{ ln}(x)\prec\text{ ln}(x)^\alpha\prec x\prec x^2\prec x^{\alpha>2}\prec \alpha^x\sim e^x\prec x^x\]
La notation est telle que $\prec$ signifie "croît moins vite que" et $\succ$ signifie "croît plus vite que".

\section{Séries divergentes à retenir}
Les séries suivantes sont divergentes et utiles pour le critère de comparaison.
\begin{itemize}
\item $\sum_{n=1}^\infty \dfrac{1}{n} = +\infty$ (série harmonique)
\item $\sum_{n=1}^\infty \dfrac{1}{\sqrt{n}} = +\infty$
\item $\sum_{n=1}^\infty \dfrac{1}{n^\alpha} = +\infty$ si $\alpha\leq 1$
\item $\sum_{n=0}^\infty r^n$ si $r\geq 1$
\end{itemize}

\section{Séries convergentes à retenir}
\begin{itemize}
\item $\sum_{k=0}^\infty \dfrac{1}{k!} = e$ et, plus généralement, $\sum_{k=0}^\infty \dfrac{\lambda^k}{k!} = e^\lambda$
\item $\sum_{k=0}^\infty \dfrac{1}{2^k} = 2$
\item $\sum_{k=1}^\infty \dfrac{1}{k^\alpha} = \zeta(\alpha)$ si $\alpha > 1$ (fonction zêta de Riemann)
\item $\sum_{k=0}^\infty (-1)^k\dfrac{1}{k!} = \dfrac{1}{e}$
\item $\sum_{k=1}^\infty (-1)^{k-1} \dfrac{1}{k} = \text{ln}(2)$ (série harmonique alternée)
\item $\sum_{k=0}^\infty r^k = \dfrac{1}{1-r}$ si et seulement si $|r|<1$ (série géométrique)
\item $\sum_{k=0}^\infty k\cdot r^{k-1} = \dfrac{1}{(1-r)^2}$ si et seulement si $|r|<1$ (série géométrique dérivée)
\end{itemize}

\section{Limites de quelques fonctions}
Toujours utiles à savoir et souvent présentes en examen ou en exercice.\\
\begin{tabular}{ |c || c | c | c| }
\hline
 & $x\to-\infty$ & $x\to0$  & $x\to+\infty$ \\
 \hline 
 $e^x$ & 0 & 1 & $+\infty$ \\  
 $\text{ln}(x)$ & $\nexists$ & $-\infty$ & $+\infty$ \\
 $1/x$ & 0 & $\nexists$ & 0 \\
 \hline
\end{tabular}

\section{Quelques suites célèbres}
Les suites dont je parle ici n'ont pas été étudiées en cours et ne sont pas au programme mais je les ai trouvées intéressantes et/ou amusantes.
\paragraph{Suite de Conway}
Cette suite consiste à énoncer le nombre de chiffres du terme précédent.  Elle est qualifiée de suite "audioactive" ou "Look and Tell sequence" en anglais.\par 
$a_0 = 1$ (terme initial)\par 
$a_1 = 11$ (une fois 1)\par
$a_2 = 21$ (deux fois 1)\par 
$a_3 = 12\ 11$\par 
$a_4 = 11\ 12\ 21$\par 
$a_5 = 31\ 22\ 11$\par 
$a_6 = 13\ 11\ 22\ 21$\par 
... et ainsi de suite

\chapter{Nombres complexes}
\section{Relation entre forme polaire et trigo}
\subsection{Formule d'Euler et dérivées}
La formule d'Euler nous donne la relation suivante :
\[e^{i\phi} = \text{cos}(\phi)+i\text{ sin}(\phi)\]
qui nous permet de passer de la forme polaire à la forme cartésienne d'un nombre complexe. La même formule est applicable pour le conjugué :
\[e^{-i\phi} = \text{cos}(\phi)-i\text{ sin}(\phi)\]
car le cosinus est pair et le cosinus impair. Si on additionne maintenant les deux formes, on a
\[e^{i\phi}+e^{-i\phi} = (\text{cos}(\phi) + i\text{ sin}(\phi)) + (\text{cos}(\phi) - i\text{ sin}(\phi)) = 2\text{ cos}(\phi)\]
De manière similaire, on a une formule pour le sinus :
\[e^{i\phi}-e^{-i\phi} = 2i\text{ sin}(\phi)\]
Il ne faut pas confondre ces égalités avec les définitions des sinus et cosinus hyperboliques, ici la puissance des exponentielles est bien complexe et non pas réelle.

\chapter{Intégration}
\section{Techniques d'intégration vues en cours}
\subsection{Relation de Chasles}
Soient $f$ une fonction continue sur un intervalle $I$ et $a,b,c$ trois réels de $I$.
\[\int^b_af(x)dx + \int^c_bf(x)dx = \int^c_af(x)dx\]
Cette égalité permet de séparer en plusieurs parties une intégrale, par exemple pour une valeur de $b$ particulière et permettant d'utiliser d'autres techniques.

\section{Techniques d'intégrations pas vues en cours}
\subsection{La technique des "2I"}
Souvent, on désigne l'intégrale que l'on étudie par $I$, d'où le nom de cette méthode.\par 
Lorsque l'on trouve une égalité pour une intégrale, c'est-à-dire deux inté\-grales qui sont égales mais ne sont pas exactement les mêmes, grâce par exemple, aux autres méthodes listées ici, on peut les additionner pour élimer des termes et simplifier l'expression. On a alors une intégrale plus facile à manipuler. Il ne faut pas oublier cependant de diviser par 2 à la fin de notre calcul, car on a additionné l'intégrale avec elle-même ($2I$).\par 
En gros, si on a $I=\int f(x)dx = \int g(x)dx$, on sait que $2I = \int f(x)+g(x)dx$, ce qui peut parfois être simplifié.

\subsection{Formule du roi}
Soit $f:[a,b]\to\R$ une fonction continue. Alors 
\[\int^a_b f(x) dx = \int^a_b f(a+b-x) dx\]
On a un résultat analogue pour les sommes : $\sum^b_{k=a} f(k) = \sum^b_{k=a} f(a+b-k)$. Il semble alors logique que les deux soient reliés, puisqu'on peut voir une intégrale comme une somme, notamment à travers la somme des rectangles sous une fonction.\par 
On prouve cette formule par un changement de base que je ne détaillerai pas ici mais dont l'intuition est que $x=a+b-u$.\par 
Il est alors intéressant de noter que $\int^{\pi/2}_0 cos(x)dx = \int^{\pi/2}_0 cos(\dfrac{\pi}{2}-x)dx = \int^{\pi/2}_0 sin(x)dx$\par 
Cette méthode peut être utile dans les cas où $a+b$ vaut une valeur remarquable, par exemple un multiple de $\pi$ dans le cas d'une intégration d'une formule trigonométrique. C'est aussi utile lorsque $a+b=0$, car on remplace $x$ par $-x$.\footnote{Ces explications sont tirées d'une vidéo par Axel Arno, \url{https://www.youtube.com/watch?v=a5wgeZ78U_A}.}

\subsection{Formule de la reine}
Pour une intégrale définie sur l'intervalle $[0,2a]$, on sait que 
\[\int^{2a}_0 f(x)dx = \int^{a}_0 f(x)+f(2a-x)dx \]
Ici, le changement de variable pour démontrer la formule est $x=2a-u$, combiné à la relation de Chasles.\par 
Contrairement à la formule du roi, celle-ci est beaucoup plus situationnelle. Elle est sensée tirer parti de la symétrie d'une fonction, mais je n'ai pas réussi à trouver d'explication satisfaisante.

\subsection{Formule du valet}
Pour une fonction $T$-périodique $f(x)$, sur un intervalle $[0,nT]$, on a
\[\int^{nT}_0 f(x)dx = n\int^T_0 f(x)dx\]
ce qui fait sens car une fonction périodique aura toujours la même valeur pour une intégrale avec un intervalle égal à sa période, si on le décale de $n$ périodes. L'intégrale de $n$ périodes est donc égale à $n$ intégrales de la période.

\chapter{Astuces et reminders pour l'examen}
\section{Contres-exemples courants}
Il est utile de tester les propositions données avec des contres exemples typiques. Pour les fonctions, on a :
\begin{itemize}
\item $f(x) = \dfrac{1}{x}$
\item $f(x) = e^x$ et son inverse, $g(x) = \text{ln}(x)$
\end{itemize}
Pour les suites :
\begin{itemize}
\item Les suites alternées, c'est-à-dire contenant $(-1)^n$
\end{itemize}

\end{document}
