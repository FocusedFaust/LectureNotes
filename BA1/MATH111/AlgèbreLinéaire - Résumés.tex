\documentclass[10pt,a4paper]{book}
\usepackage[utf8]{inputenc}
\usepackage{amsmath}
\usepackage{amsfonts}
\usepackage{amssymb}
\usepackage{multicol}
\usepackage{hyperref}
\usepackage[ruled, lined, longend]{algorithm2e}
\usepackage[shortlabels]{enumitem}
\usepackage{textcomp}

\setlength{\parindent}{20pt}

\newcommand{\R}{\mathbb{R}}
\newcommand{\N}{\mathbb{N}}
\newcommand{\Z}{\mathbb{Z}}
\newcommand{\F}{\mathbb{F}}
\newcommand{\ind}{\hspace*{\parindent}}

\title{Algèbre Linéaire - Notes et Résumés}
\author{Faustine Flicoteaux}
\date{Semestre d'automne 2024}

\begin{document}
\maketitle
\tableofcontents
\newpage


\section*{Introduction}
Ce qui suit est principalement mes propres notes et explications. Évidemment, tout ça ne me vient pas par l'opération du Saint-Esprit, mais du cours de mon professeur, Prof. Jérôme Scherer, et de vidéos et cours en ligne (merci en particulier à 3Blue1Brown).\par 
J'explique principalement des concepts que je ne comprends pas totalement, ce document est donc à prendre avec des pincettes. Si vous remarquez une erreur (mathématique ou de langue), n'hésitez pas à m'en faire part à mon adresse e-mail \texttt{\href{mailto:faustine.flicoteaux@epfl.ch}{faustine.flicoteaux@epfl.ch}}.\par 
Le dernier fichier \LaTeX et le pdf correspondant peuvent être trouvés sur mon dépôt GitHub \url{https://github.com/FocusedFaust/LectureNotes}.

\chapter{Matrices}
\section{Opérations du déterminant}
La multiplication est préservée pour le déterminant, donc $det(A)\cdot det(B) = det(A\cdot B)$. En particulier, $det(A^{-1}) = (det(A))^{-1}$. Par contre, il n'y a pas de formule pour le déterminant d'une addition de matrices, car le déterminant n'est pas linéaire.\par 
Échanger deux lignes ou colonnes revient à multiplier le déterminant par $-1$.\par 
Multiplier une ligne ou une colonne par un scalaire $a$ revient à multiplier le déterminant par ce même scalaire.

\section{Valeurs propres facilement repérables}
Certaines valeurs propres peuvent être facilement repérables, sans avoir à passer par le polynôme caractéristique, ce qui nous évite de perdre pas mal de temps quand on a des grandes matrices. \par 
Il est utile de noter que pour une matrice $A$ diagonalisable, $A^T$ a les mêmes valeurs propres.
\begin{itemize}
\item Si deux colonnes (ou plus) sont linéairement dépendantes, alors le $Ker$ de la matrice n'est pas nul et 0 est une valeur propre (et donc la matrice $A$ n'est pas inversible car son déterminant vaut 0). 
\item Les matrices triangulaires (supérieures ou inférieures) on comme valeurs propres les coefficients placés dans la diagonale.
\item Si une des colonnes est vide sauf en $a_{ii}$ avec $i$ l'index de la colonne, alors le coefficient $(a_{ii})$ est une valeur propre. En effet, si on calcule $A-(a_{ii})\cdot I_n$, on se retrouve avec une colonne nulle. On a donc forcément $dim(Ker(A)) > 0$ et donc $(a_{ii})$ est une valeur propre. 
\item Si la somme des coefficients des lignes est la même pour toutes les lignes ou si la somme des coefficients des colonnes est la même pour toutes les colonnes, alors cette somme est valeur propre de la matrice (prouvé en exercice).
\item Si $A$ est diagonalisable, alors la trace de $A$ est égale à la somme des valeurs propres de $A$. Ainsi, si il nous manque une seule valeur propre (attention à la multiplicité), on peut la calculer à partir de la trace.
\end{itemize}

\section{Factorisation QR}
Le but de la factorisation $QR$ est de trouver une décomposition d'une matrice $A$ telle que $Q\cdot R=A$ avec la matrice $Q$ qui soit orthogonale et la matrice $R$ qui soit triangulaire supérieure.\par 
Pour trouver la matrice $Q$, on utilise la méthode de Gram-Schmidt sur les colonnes de $A$, pour en obtenir une base orthogonale de l'espace-colonne. On divise ensuite les vecteurs obtenus pour les normaliser.\par 
La matrice $R$ correspond aux colonnes de $A$ exprimés dans la base de $Q$. Puisque $Q\cdot R=A$ et que $Q$ est orthonormée ($Q^{-1}=Q^T$), on peut alors calculer $R=Q^T\cdot Q\cdot R=Q^T\cdot A$. \par
Sinon, on peut remarquer que la matrice $R$ correspond à l'inverse (donc signe inverse) des opérations élémentaires effectuées lors du "Gram-Schmidt-age" de $A$. On multiplie ensuite chaque ligne de $R$ ainsi obtenu par la norme des vecteurs de $Q$ avant normalisation (méthode certes plus rapide mais plus prône aux erreurs).

\chapter{Bases et matrices}

\section{Représentation dans une base}
Soient un espace vectoriel $V$, un vecteur $\vec{v} = \begin{pmatrix} v_1 \\ \vdots \\ v_n \end{pmatrix}$ et une base $\mathcal{B}$ de $V$. Pour trouver les coordonnées de $\vec{v}$ dans la base $\mathcal{B}$, il faut résoudre le système $B\vec{x} = \vec{v}$ avec $B$ la matrice qui représente la base $\mathcal{B}$ et $\vec{x} = \begin{pmatrix} x_1 \\ \vdots \\ x_n \end{pmatrix} = (\vec{v})_\mathcal{B}$, c'est à dire le vecteur $\vec{v}$ exprimé selon la base $\mathcal{B}$.

\section{Changements de bases}
Soit V un espace vectoriel. On peut représenter un vecteur de cet espace en fonction d'une base arbitraire. Lorsqu'on change de base, la représentation sera alors différente. C'est-à-dire qu'un vecteur $\vec{b}$ n'a pas les mêmes coordonnées en fonction de la base dans laquelle il est représenté. \par 
Soit une base $\mathcal{B}=(\vec{b_1}, \vec{b_2})$. Un vecteur $(x)_{\mathcal{B}}=(x_1,x_2)$ signifie $x_1\cdot \vec{b_1}+x_2\cdot \vec{b_2}$. Pour trouver les valeurs qu'auraient $x_1$ et $x_2$ dans une autre base, par exemple $\mathcal{C}=(\vec{c_1}, \vec{c_2})$ on utilise une matrice de changement de base, selon l'égalité suivante: 
\[(Id)^{\mathcal{C}}_{\mathcal{B}} (x)_{\mathcal{B}} = (x)_{\mathcal{C}}\]
La matrice $(Id)^{\mathcal{C}}_{\mathcal{B}}$ est la matrice de changement de base de $\mathcal{B}$ vers $\mathcal{C}$. Comme dit Prof. Scherer : "\textit{Elle mange des vecteurs en base $\mathcal{B}$ pour donner des vecteurs en base $\mathcal{C}$}". \par 
Pour la trouver, on écrit la matrice augmentée $[\vec{c_1},\vec{c_2}\ |\ \vec{b_1},\vec{b_2}]$ puis on l'échelonne et la réduit. Cela permet d'exprimer les coordonnées des vecteurs de la base $\mathcal{B}$ comme combinaison linéaire des vecteurs de la base $\mathcal{C}$, c'est-à-dire que $(Id)^{\mathcal{C}}_{\mathcal{B}} = ((\vec{b_1})_{\mathcal{C}},(\vec{b_2})_{\mathcal{C}})$.\par 
On remarque alors que la matrice de changement de base est une transformation linéaire (et même un endomorphisme) qui permet de projeter les vecteurs d'une base sur les vecteurs d'une autre base. Tout vecteur autre sera alors projeté sur sa représentation en une autre base. C'est-à-dire qu'un vecteur $(a,b)_\mathcal{C}$ sera projeté sur $(a,b)_\mathcal{B}$ (mêmes coordonnées mais vecteurs différents) alors qu'un vecteur $(a,b)_\mathcal{B}$ sera projeté sur $(c,d)_\mathcal{C}$(même vecteur mais différentes coordonnées). 

\paragraph{Changement de base inverse}
Puisque le changement de base peut s'appa\-renter à une transformation linéaire, le changement inverse est la transformation linéaire inverse. $(Id)^{\mathcal{B}}_{\mathcal{C}} = ((Id)^{\mathcal{C}}_{\mathcal{B}})^{-1}$

\paragraph{Astuce}
Il y a un moyen plus rapide de calculer une matrice de changement de base : $(\vec{c_1},\vec{c_2})^{-1}\cdot (\vec{b_1},\vec{b_2}) = (Id)^{\mathcal{C}}_{\mathcal{B}}$\footnote{\textit{Cette méthode n'a pas été démontrée par le prof (et encore moins par moi), c'est donc à utiliser à vos risques et périls.}}

\paragraph{Cas de la base canonique}
La base canonique se note $\mathcal{C}an = (e_1,...,e_n)$ avec $e_i$ des vecteurs dont les coordonnées valent 0 sauf quand l'index est égal à $i$, ou elle vaut 1. C'est la base la plus "naturelle" pour nous autres, pauvres mortels, pour se représenter un espace vectoriel. La matrice de changement de base est alors assez intuitive : $(Id)^{\mathcal{C}an}_{\mathcal{B}}$ est la matrice ayant pour colonnes les vecteurs de la base $\mathcal{B}$.

\paragraph{Astuce en examen}
Pour une question à choix multiples de type "\textit{Soit deux bases données, quelle est la matrice de changement de base ?}" il n'est pas forcément utile de calculer la matrice entière. Il ne faut pas non plus commencer par le premier vecteur. Il est plus rapide de trouver pour les matrice proposée une colonne qui est différente pour chaque proposition. Ensuite, on peut calculer $(\vec{b_i})_\mathcal{C}$ correspondant.\par 
Le temps est compté en examen. Il faut donc essayer d'éliminer le plus de réponses possibles en le moins de calculs possibles.

\section{Matrice d'une application selon une base}
Il peut être demandé de construire la matrice d'une application linéaire $T$ mais dans une base spécifique, disons $\mathcal{B}=(\vec{b_1},...,\vec{b_n})$. Dans ce cas la matrice sera $M = [(T(\vec{b_1}))_\mathcal{B}\ ...\ (T(\vec{b_n}))_\mathcal{B}]$.

\paragraph{Astuce en examen}
Lors du calcul de la matrice d'une transformation selon une base, il faut bien penser à exprimer le résultat de chaque $T(\vec{b_i})$ selon la base demandée.

\section{Matrice d'une transformation linéaire}
Soit $T:V\to W$ une transformation linéaire de $V$ vers $W$. Pour construire la matrice représentant cette transformation, on choisit une base de $V$ et une base de $W$ (habituellement des bases canoniques). Les colonnes de la matrice $A$ de $T$ sont les images des vecteurs de la base de $V$ exprimés dans la base de $W$. \par 
On voit dans le cas d'un endomorphisme ($T:V_\mathcal{B}\to V_\mathcal{C}$) ou l'on choisit deux bases $\mathcal{B}$ et $\mathcal{C}$ différentes pour $V$, que la matrice sera les vecteurs de $\mathcal{B}$ dans la base $\mathcal{C}$.

\chapter{Corps finis}
\paragraph{Caractéristique et cardinalité}
La caractéristique d'un corps fini $K$ est le plus petit entier non nul $n$ tel que $n\cdot 1_K = 0$. On le note $carK$. Cette caractéristique est un nombre premier.\par 
Soit $K$ un corps fini et $p=carK$. Alors il existe $n$ tel que $K$ a $p^n$ éléments. Ce nombre est la cardinalité de $K$.

\section{Corps à partir de nombres premiers}
Les corps finis avec un nombre premier d'élément sont de prime abord relativement simples (je dis relativement parce qu'on fait quand même de la théorie des corps). Un corps $\F_p$ est un corps à $p$ éléments $\{0,1,...,p-1\}$. Toutes les opérations se font \textit{modulo} $p$, c'est à dire qu'on ne considère que le reste de la division entière par $p$.

\paragraph{Opposé}
Chaque nombre $x\neq 0$ admet un opposé $-x$ tel que $x+(-x)=0$.\par
Par exemple, dans le corps $\F_7$, l'opposé de 3 est 4 car $3+4=7=0$.

\paragraph{Inverse}
Chaque nombre $x\neq 0$ admet un inverse $x^{-1}$ tel que $x\cdot x^{-1}=1$.\par
Par exemple, dans le corps $\F_7$, l'inverse de 3 est 5 car $3\cdot 5=15=1$.

\section{Corps à partir de polynômes}
Lorsque l'on veut former un corps dont le nombre d'éléments n'est pas premier, on se rend compte que l'addition et la multiplication ne permettent pas d'en faire un groupe abélien. Pour former un corps, au lieu des entiers, on utilise des polynômes $\F_p[t]$ comme éléments et la division euclidienne est remplacée par la division polynomiale.\par 
Pour former un corps, il nous faut un polynôme $p(t)$ unitaire irréductible de degré $n$ dans $\F_p[t]$. Ensuite, on considère l'ensemble de tous les restes de division par $p(t)$. Il y en a $p^n$.

\chapter{Astuces et reminders pour l'examen}
\section{Calcul de l'inverse d'une matrice}
Cette méthode m'a été transmise par mon incroyable professeur de l'année dernière, M. Ischi, et s'est avérée très utile. Elle marche principalement sur les matrices $2\times 2$ et $3\times 3$ (au-delà, les calculs ne sont pas plus simples que la formule "officielle"). Pour une matrice $A$, on calcule les coefficients de son inverse tels que 
\[(a^{-1})_{ij} = \dfrac{1}{detA} det((A^T)_{ij})\]
où $(A^T)_{ij}$ est la matrice transposée à laquelle on a enlevé la ligne $i$ et la colonne $j$. C'est-à-dire que chaque coefficient de la matrice inverse est le déterminant de $(A^T)_{ij}$, divisé par le déterminant de $A$. Attention cependant, parce que chaque coefficient est multiplié par un signe selon sa ligne et sa colonne, comme suit :
\[\begin{pmatrix}
 + & - & +\\
 - & + & -\\
 + & - & +
\end{pmatrix}\]
Attention, Il faut bien utiliser la transposée de la matrice à inverser, et non la matrice elle-même.

\section{Contres-exemples courants}
Il est utile de tester les propositions données avec des contres exemples typiques. Pour les matrices, on a :
\begin{itemize}
\item $\big(\begin{smallmatrix} 
1 & 0\\
0 & 0
\end{smallmatrix}\big)$, 
$\big(\begin{smallmatrix} 
0 & 1\\
0 & 0
\end{smallmatrix}\big)$, 
$\big(\begin{smallmatrix} 
0 & 0\\
1 & 0
\end{smallmatrix}\big)$ et  
$\big(\begin{smallmatrix} 
0 & 0\\
0 & 1
\end{smallmatrix}\big)$, notamment pour l'inversibilité.
\item La matrice identité
\item La matrice nulle
\item Une matrice qui ne soit pas carrée (pour l'inversibilité). Effectivement, il ne faut pas supposer qu'une matrice soit inversible, à moins que ça ne soit explicitement dit dans l'énoncé.
\end{itemize}

\section{Erreurs de nomenclature}
Attention, orthonormé $\neq$ orthogonal. Dans les questions portant sur des bases orthonormées, il faut veiller à éliminer d'office les bases qui ne le sont pas.\par 
Ce n'est pas une \textbf{unique} solution si la solution est un sous-espace vectoriel.

\end{document}